% This is the Fancy Tutorial created by Gabriel Yin
% Southern Methodist University, TX, USA, 75206
% Contact: gyin@smu.edu.
% All Rights Reserved since 2018
% Any issue report or comment is SUPER welcome to be sent to me!!

% This tutorial is designated for LaTeX first-runners in Windows system.

% If you are using MacOX or Linux systems, you can also learn a lot from this tutorial!! But I EXTREMELY recommend the following link to you (remember to uncomment)

% https://github.com/xuhdev/vim-latex-live-preview

% This is the LaTeX compiling plug-in in Vim, which is a very powerful text/code writing and documenting software, and it is OPEN-SOURCE! But that's more or less advanced usage.

% Now let's get started.

% LaTeX uses backslash, i.e. "\", to indicate a built-in function.
% The format of this LaTeX file is IEEE Transaction Journal, i.e. \documentclass[journal]{IEEEtran}
\documentclass[journal]{IEEEtran}

%%%%%%%%%%%%%%%%%
%%% Head File (Preambles) %%%
%%%%%%%%%%%%%%%%%

% if you think they are tooooooo lengthy, you can put all of the preambles into ONE another "preamble.tex" file, and use command: \include{preamble} to replace the following tedious lists.

\usepackage{amsmath}					% amsmath is IMPORTANT
\DeclareMathOperator*{\argmax}{arg\,max}
\DeclareMathOperator*{\argmin}{arg\,min}
\usepackage{amssymb}					% amssymb is IMPORTANT
\usepackage{amsthm}					% amsthm is IMPORTANT
\usepackage{amsfonts}					% amsfonts is IMPORTANT
\usepackage{mathrsfs}					% mathrsfs is IMPORTANT
\usepackage{mathtools}					% mathtools is IMPORTANT			
\usepackage{mathdots}					% For dot sign
\usepackage{fancyhdr}					% For head franchise 
\usepackage{longtable}					% For long table
\usepackage{xcolor}						% For colorful expression
\usepackage{multirow, makecell}			% For fancy table cell and multi-row tables
\usepackage{booktabs}					% For convenient booktabs
\usepackage{siunitx}					% For miscellaneous math symbols
\usepackage{nomencl}					% For nomenclature
\usepackage[short]{optidef}				% For definition marks				
\usepackage[utf8x]{inputenc}				% For unicode character
\usepackage{kantlipsum}					% For fake Latin text in Kant style
\usepackage[pdftex]{graphicx} 			% For adding figures (IMPORTANT)
\usepackage{cite}						% For citations (IMPORTANT)
\usepackage{pgfplots}					% For modifying plots and generating figures itself
\usepackage{epstopdf}					% For transferring EPS file to PDF file and adding to LaTeX
\usepackage{listings}					% For listing any code in LaTeX
\usepackage{enumerate}					% For making list (IMPORTANT)
\usepackage{tabularx}					% For making fancy tables (IMPORTANT)
\usepackage[normalem]{ulem}				% For highlighting and emphasis
\usepackage{algorithm, algorithmicx}		% For presenting algorithms
\usepackage[flushleft]{threeparttable}		% For miscellaneous alignments
\usepackage{changepage}				% For \adjustwidth command for writing pseudocodes

% Those with IMPORTANT marks are critical for researchers to write a paper with many mathematical symbols and other fancy and convenient styles.

% User-defined Functions
\newtheorem{definition}{Definition}
\newtheorem{remark}{Remark}
\newtheorem{theorem}{Theorem}
\newtheorem{lemma}[theorem]{Lemma}
\newtheorem{corollary}[theorem]{Corollary}
\newtheorem{proposition}[theorem]{Proposition}
\newtheorem{claim}[theorem]{Claim}
\newtheorem{observation}{Observation}
\newtheorem{fact}{Fact}
\newtheorem{example}{Example}
\newtheorem{exercise}{Exercise}
\newtheorem{notation}{Notation}

% Define the Graph Directory Path
\graphicspath{ {/Users/gabriel79/Desktop/} }

% Set Display Length
\setlength{\abovedisplayskip}{0.1pt}
\setlength{\belowdisplayskip}{0.1pt}

% Define the cells to be used in Tables.
% If you do not want to use cells now, delete all of them now.
\def\InvB{\makecell{Trans. Investment\\ budget [M\$]}}
\def\TotalCost{\makecell{Total\\ cost [M\$]}}
\def\IterNum{\makecell{Iteration\\ number}}
\def\WallTime{\makecell{Elapsed\\ time [s]}}
\def\WallTimeh{\makecell{Elapsed\\ time [s]}}
\def\TranExp{\makecell{Transmission\\ expansion}}
\def\GenExp{\makecell{Generation \\ $\text{expansion}^*$}}
\def\LoL{\makecell{Loss of\\ load rate}}
\def\one{\makecell{{1}}}
\def\two{\makecell{{2}}}
\def\three{\makecell{{3}}}

\def\TCo{\makecell{{1-5,} \\ {4-6}}}
\def\GCo{\makecell{{1, 2,} \\ {3, 4, 5}}}
\def\TcoC{\makecell{{99.16}}}
\def\TcoD{\makecell{{104.57}}}
\def\InoC{\makecell{{7}}}
\def\InoD{\makecell{{8}}}
\def\WtoC{\makecell{{9.2}}}
\def\WtoD{\makecell{{25.4}}}
\def\LoLoC{\makecell{{6.2\%}}}
\def\LoLoD{\makecell{{7.3\%}}}

\def\TCt{\makecell{{2-3, 2-5} \\ {3-5, 4-6}}}
\def\TDt{\makecell{{1-5, 2-4} \\ {2-5, 2-6}}}
\def\GCt{\makecell{{2, 4,} \\ {5, 6}}}
\def\GDt{\makecell{{1, 2,} \\ {4, 6}}}
\def\TctC{\makecell{{67.19}}}
\def\TctD{\makecell{{63.57}}}
\def\IntC{\makecell{{13}}}
\def\IntD{\makecell{{11}}}
\def\WttC{\makecell{{30.1}}}
\def\WttD{\makecell{{166.5}}}
\def\LoLtC{\makecell{{6.0\%}}}
\def\LoLtD{\makecell{{5.2\%}}}

\def\TCth{\makecell{{1-2, 2-3,} \\ {2-4, 2-5} \\ {4-6}}}
\def\TDth{\makecell{{1-5, 2-3,} \\ {2-4, 3-4} \\ {4-6}}}
\def\GCth{\makecell{{1, 2, 5}}}
\def\GDth{\makecell{{2, 4, 5}}}
\def\TcthC{\makecell{{56.51}}}
\def\TcthD{\makecell{{59.71}}}
\def\InthC{\makecell{{12}}}
\def\InthD{\makecell{{9}}}
\def\WtthC{\makecell{{27.3}}}
\def\WtthD{\makecell{{110.2}}}
\def\LoLthC{\makecell{{2.1\%}}}
\def\LoLthD{\makecell{{2.1\%}}}

\def\Type{\makecell{Types}}
\def\Wind{\makecell{Wind}}
\def\Conv{\makecell{Thermal}}
\def\OpeC{\makecell{{Operational} \\ {cost} \\ {[\$/MWh]}}}
\def\MaxC{\makecell{{Maximum} \\ {capacity} \\ {[MW]}}}
\def\InvC{\makecell{{Investment} \\ {cost} \\ {[M\$]}}}
\def\OpeCw{\makecell{0}}
\def\OpeCc{\makecell{63}}
\def\MaxCw{\makecell{300}}
\def\MaxCc{\makecell{300}}
\def\InvCw{\makecell{20}}
\def\InvCc{\makecell{8}}

\def\GInvB{\makecell{Gen. Investment\\ budget [M\$]}}

% Here ends the Head File (Preambles).

%%%%%%%%%%%%%%%%%
%%%%%% Main File %%%%%%
%%%%%%%%%%%%%%%%%

% The Beginning of the Paper
\begin{document}

\title{A Fancy Tutorial for Writing IEEE Transaction Journal Paper in \LaTeX}

\author{Gabriel~Yin,~\IEEEmembership{Student~Member,~IEEE,} and~Gabriel~Yin,~\IEEEmembership{Member~IEEE}

\thanks{The work of G. Yin and G. Yin is partly supported by Southern Methodist University, U.S.A,  under Project DOD-PENTAGON-0001234. (seriously, NO)} 
\thanks{G. Yin and G. Yin are with the Lyle School of Engineering, Southern Methodist University, TX, U.S.A. (emails: \{gyin, gyin\}@smu.edu).}}

\markboth{IEEE TRANSACTION ON ...}
{G. Yin\MakeLowercase{\textit{et al.}}: }
\maketitle

\begin{abstract}
Type your Abstract.
\end{abstract}

\begin{IEEEkeywords}
Type your Keywords.
\end{IEEEkeywords}

\IEEEpeerreviewmaketitle

\section{Introduction}
\IEEEPARstart{H}{ere} is your intro. If you are going to use this Big ``H" head, please make efforts to finish this paragraph in three lines. Yes, like what I am doing now.\\
\indent New Paragraph \dots\\
\indent if you like to enumerate,
\begin{enumerate}[1)] 
\item First item.
\item Second item.
\end{enumerate}

You can change the type of enumerate from 1) to i), or a), or just dot (delete the [1)].). And also you can use other functions like typing math within the enumeration. 
\begin{enumerate}[a)] 
\item First item: $\mathbb{G}\alpha b\text{riel}^{is} \mathcal{A} \ N_{ice} \ G_{uy}$.
\item Second item.
\end{enumerate}

\texttt{Itemize} is another type of enumerate. It does not requires any package, but can only use dots to present the items.
\begin{itemize}
\item First item.
\item Second item.
\end{itemize}
The rest of the paper is fivefold: BlahBlahBlah

% New section
\section{Mathematical Formulation and BlahBlahBlah}
% Add your nomenclature. I keep my nomenclature for you guys for reference.
% You can also put the nomenclature at the beginning.

% And you might have noticed, it is a \subsection within a section.
\subsection{Nomenclature}
The major symbol notations used in this paper are presented here. Note that vectors and matrices are represented by bold lower-case and upper-case characters, respectively. Other notations are given throughout the paper if necessary.\\
\noindent \emph{1) \ Sets}:
\addcontentsline{toc}{section}{Nomenclature}
\begin{IEEEdescription}[\IEEEusemathlabelsep\IEEEsetlabelwidth{$V_1,V_2,V_3$}]
\item[$G \, / \, R \, / \, L$] Conventional generator / transmission line.
\item[$S \, / \, V$] Sending/receiving bus of $\ell$th transmission line.
\item[$C$] Generator\&bus mapping of $g$th or $r$th generator
\item[$X^G \, / \, X^R$] Candidate traditional/renewable generator.
\item[$X^L$] Candidate transmission line.
\item[$T$] Planning horizon.
\end{IEEEdescription}
\emph{2) \ Indices}:
\addcontentsline{toc}{section}{Nomenclature}
\begin{IEEEdescription}[\IEEEusemathlabelsep\IEEEsetlabelwidth{$V_1,V_2,V_3$}]
\item[$g \, / \, r \, / \, \ell$] Conventional generator / renewable generator / transmission line.
\item[$d$] Demand.
\item[$n$] Bus.
\item[$\omega$] Scenario.
\item[$k$] Iteration counter for C\&CG algorithm.
\item[$o$] Iteration counter for L-shaped algorithm.
\item[$t$] Planning year.
\end{IEEEdescription}
\emph{3) \ Parameters}:
\addcontentsline{toc}{section}{Nomenclature}
\begin{IEEEdescription}[\IEEEusemathlabelsep\IEEEsetlabelwidth{$V_1,V_2,V_3$}]
\item[$IC_{g,t}^G/IC_{r,t}^R$] Investment cost for $g$th conventional / $r$th renewable generator in $t$th year [\$].
\item[$IC_{\ell,t}^L$] Investment cost for $\ell$th transmission line in $t$th year [\$].
\item[$OC_{g,t}$] Operating cost for $g$th generator in $t$th year [\$/MWh].
\item[$PC_{d,t}$] Load-shedding cost for unserved load in $d$th demand in $t$th year [\$/MWh].
\item[$B_t^G \, / \, B_t^L$] Investment budget for generator/transmission line in $t$th year [\$].
\item[$K_t^G \, / \, K_t^L$] Contingency criteria for generator/transmission line in $t$th year.
\item[$FL_{\ell,t}$] Flow capacity of $\ell$th transmission line in $t$th year [MW].
\item[$\overline{PL}^G_{g,t}$] Capacity limit of $g$th conventional generator in $t$th year [ MW].
\item[$\underline{\Delta}_{n,t} \, / \, \overline{\Delta}_{n,t}$] Angle limit of voltage angle at $n$th bus in $t$th year [rad].
\end{IEEEdescription}
\emph{4) \ Uncertain Parameters}:
\begin{IEEEdescription}[\IEEEusemathlabelsep\IEEEsetlabelwidth{$V_1,V_2,V_3$}]
\item[$\overline{PL}^R_{r,t}$] Capacity limit of $r$th renewable generator in $t$th year  [MW].
\item[$P_{d,t}$] Demand of $d$th load in $t$th year [MW].
\end{IEEEdescription}
\emph{5) \ Variables}:
\addcontentsline{toc}{section}{Nomenclature}
\begin{IEEEdescription}[\IEEEusemathlabelsep\IEEEsetlabelwidth{$V_1,V_2,V_3$}]
\item[$x_{g,t}^G/x_{r,t}^R$] Binary expansion decision for $g$th traditional / $r$th renewable generator in $t$th year. $x=1$ means built; $x=0$ otherwise.
\item[$x_{\ell,t}^L$] Binary expansion decision for $\ell$th transmission line. $x=1$ means built; $x=0$ otherwise.
\item[$a_{g,t}^G$] Binary outage indicator for $g$th conventional generator. $a=1$ means outage; $a=0$ otherwise.
\item[$a_{\ell,t}^L$] Binary outage indicator for $\ell$th transmission line. $a=1$ means outage; $a=0$ otherwise.
\item[$p^G_{g,t}/p^R_{r,t}$] Scheduled power output of $g$th conventional / $r$th renewable generator in $t$th year [MW].
\item[$r_{d,t}$] Load shedding in $d$th demand in $t$th year [MW].
\item[$f_{\ell,t}$] Power flow through $\ell$th transmission line in $t$th year [MW].
\item[$\delta_{n,t}$] Voltage angle at $n$th bus in $t$th year [rad].
\item[$\eta$] Dummy variable formulating a relaxed lower bound in the master problem.
\end{IEEEdescription}
% Your subsection starts here.
\subsection{Problem Description}
Here it flies ...

% How to type your equation:

\begin{align*} 
% * means no numbering. You can add your own numbering by using \tag{1a}, \tag{1b} ... 
% & means the same alignment.
\makebox[0cm][r]~&\sum_{g}^{X^G}IC_{g,t}^Gx_{g,t}^G+\sum_{g}^{X^R}IC_{r,t}^Rx_{r,t}^R\leq B_t^G, \ \ \forall t, \tag{1a}\\
& \sum_{\ell}^{X^L}IC_{\ell,t}^Lx_{\ell,t}^L\leq B_t^L, \ \ \forall t, \tag{1b} 
\end{align*}
Here is an example of auto-numbered align environment:
\begin{align} 
\makebox[0cm][r]~&\sum_{g}^{X^G}IC_{g,t}^Gx_{g,t}^G+\sum_{g}^{X^R}IC_{r,t}^Rx_{r,t}^R\leq B_t^G, \ \ \forall t, \\
& \sum_{\ell}^{X^L}IC_{\ell,t}^Lx_{\ell,t}^L\leq B_t^L, \ \ \forall t,  
\end{align}
% After one equation, you don't want your paragraph to be indented right??
\noindent Here it flies ...\\
\indent Your new paragraph ...

% So ends this section
\section{Solution Strategy}
BlahBlahBlah...
\subsection{Hi}
BlahBlahBlah...
\subsection{Hey}
BlahBlahBlah...
\subsection{Algorithm (or so-called Pseudocode)}
One paragon algorithm presentation is illustrated in Algorithm 1. You can find other formats in Google.
\begin{algorithm}[H]
% Exercise: Try to remove [H] and see what's gonna happen :P
% Answer: [H] means stick to the position. 
	\caption{Sample Algorithm Presentation}\label{alg:SAP}
	\begin{algorithmic}
		\State \emph{Initialization}: Set $o\leftarrow0$, $LB_o\leftarrow-\infty$, $UB_o\leftarrow \infty$, $\mathcal{O}_o=\emptyset$. 
		\State \textbf{Step 1.} $o\leftarrow o+1$. 
		\State \textbf{for} $d\in D$ \textbf{do}:
		\begin{adjustwidth}{0.5cm}{}
		\textbf{Step 2.} Write normal texts or $\mathbb{MATH} \ e^q_uations$.
		\State \textbf{Step 3.} You can even use align or equation enviroments:
\begin{align*}
& \alpha = \frac{\log 4^7}{\min_a 6a}; & \boldsymbol{\beta} = -\sum_{\omega}\text{Good}_{\omega}\bold{H}_2^\top\boldsymbol{\gamma}.
\end{align*}
BlahBlahBlah\dots \end{adjustwidth}
\textbf{end for}
		\State \textbf{Step 4.} Update $UB_o = \min\{UB_{o-1}, \text{whatever}\}$ and $LB_o = \eta^o$. If $\left|\frac{UB_o-LB_o}{UB_o}\right|\leq\varepsilon$, then terminate, declare optimality and report $(\bold{x}^{o}, \bold{y}^{o})$.
		\State \textbf{Step 5.} If you are an optimization guy like me, you might be familiar with:
\begin{align*}
M = \min_{\bold{x}, \bold{y}} \bold{c}_1^\top \bold{x}+\eta, 
\end{align*}
subject to
\begin{align*}
& \bold{H}_1^\top\bold{x}+\bold{K}_1^\top\bold{y}= \bold{r}_1, \\
& \bold{A}_1^\top\bold{x}\leq \bold{b}_1, \\
& \alpha_d^j+(\boldsymbol{\beta}_d^j)^\top\bold{y}\leq \eta/v_d, \ \ \forall j\in \mathcal{O}_o, \ \forall d\in D.
\end{align*}
Let $(\bold{x}^{o+1}, \bold{y}^{o+1}, \eta^{o+1})$ denote the optimal primal solution.\\
\textbf{goto Step 1.} 
	\end{algorithmic}
\end{algorithm}
Actually, you can make any pseudocode you want fancy with \texttt{adjustwidth} and \texttt{$\backslash$textbf}.\\
\indent You can even make more subsubsubsubsubsubsection
\subsubsection{Go Go Go}
BlahBlahBlah...
\subsubsection{Go Go Go}
BlahBlahBlah...

% How to add a figure: (Please uncomment them when using)

%\begin{figure}
%\centering
%\caption{The name of the figure}
%\includegraphics[width=9cm,height=16cm,keepaspectratio]{Your Figure Filename}
%\end{figure}

% Here goes the most important part: Case Studies!!!
\section{Illustrative Example and Case Studies}

% How to add a table:
\begin{table}[H] 
\caption{Results for Testing Two Methods}
\centering
\setlength\tabcolsep{3.5pt}
  \begin{tabular}{lllllll}
    \toprule
    &
      \multicolumn{3}{c}{Silver Bullet Method} &
      \multicolumn{3}{c}{Werewolf Method} \\
 	\InvB & \one & \two & \three & \one & \two & \three\\ % They are cells!!!! see the preambles to check the cell definitions!
            \midrule
    \TranExp & \TCo & \TCt & \TCth & \TCo & \TDt & \TDth \\[0.22cm].  
    \GenExp & \GCo & \GCt & \GCth & \GCo & \GDt & \GDth \\[0.22cm]
    \TotalCost & \TcoC & \TctC & \TcthC & \TcoD & \TctD & \TcthD\\[0.22cm]
    \LoL & \LoLoC & \LoLtC & \LoLthC & \LoLoD & \LoLtD & \LoLthD \\[0.22cm] 
    \IterNum & \InoC & \IntC & \InthC & \InoD & \IntD & \InthD\\[0.22cm]
    \WallTime & \WtoC &\WttC &\WtthC & \WtoD &\WttD & \WtthD \\
    \bottomrule
  \end{tabular}
  % How to add a tablenotes:
  \begin{tablenotes}
      \small
      \item $^*$Number shown in this row indicates the number of bus in which the generator is proposed to be built, and the same below.
    \end{tablenotes}
\end{table}
% You can type your number yourself freely without using cells. But using cells makes it fancy according to my habits :)

% Your subsection
\subsection{IEEE 118-bus System}
BlahBlahBlah...

% You are approaching the end of the Paper!!! Huuuuuuuray!!!
\section{Conclusion}
Using my tutorial file can solve most of the problems in research writings. If you have any concern, just contact me without hesitation!

% Here is the appendix
\appendices
\section{McCormick Relaxation}
BlahBlahBlah\dots
\section{Semi-convex Jensen's Inequality}
BlahBlahBlah\dots
% here is the acknowledgement
\section*{Acknowledgment}
The authors would like to thank Gabriel Yin for coaching them how to use \LaTeX. Any issue report or comment is SUPER welcome to be sent to \textcolor{blue}{\underline{gyin@smu.edu}}.

% Don't be bothered with this part =_=!!
\ifCLASSOPTIONcaptionsoff
  \newpage
\fi

% Here goes your Bibtex References!! Uncomment them before using!
% \bibliographystyle{ieeetr}         
% \bibliography{YourBibFileName}

% Here ends the Paper!! Go back to sleep!!

\end{document}